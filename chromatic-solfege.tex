
%\documentclass[twoside,a4paper]{article}
\documentclass[a4paper]{article}
\usepackage[a4paper]{geometry}
% \usepackage{showframe}

%\usepackage[T1]{fontenc}
%\usepackage{accanthis}
\usepackage{librebaskerville}
\usepackage[T1]{fontenc}

\usepackage{graphicx}
\usepackage{subcaption}
\date{}
\graphicspath{ {ly-generated/} {ly-manual/} }
\title{All about Polyrhythm}
\author{Atsushi Oka}


\usepackage{titlesec}
\setcounter{secnumdepth}{3}
\titleformat{\paragraph}
{\normalfont\normalsize\bfseries\slshape\footnotesize}{\theparagraph}{1em}{}
\titlespacing*{\paragraph}
{-1pt}{3.25ex plus 1ex minus .2ex}{-.5ex plus .2ex}


% PATCHING TITLESEC BUG 
% https://tex.stackexchange.com/questions/299969/

\usepackage{etoolbox}
\makeatletter
\patchcmd{\ttlh@hang}{\parindent\z@}{\parindent\z@\leavevmode}{}{}
\patchcmd{\ttlh@hang}{\noindent}{}{}{}
\makeatother

\begin{document}


\maketitle
\tableofcontents

\begin{abstract}
	In this paper, we will discuss about polyrhythms.
\end{abstract}

~\newline

\def \quotes [#1] #2 {%
	\begin{quote}
		\itshape\small\raggedleft
		\parbox{250pt} {
			#2
			\begin{flushright}
				\small{--- #1}
			\end{flushright}
		}
	\end{quote}
	~\newline
}

\quotes[Robert C. Martin, Clean Code] {
	The only way to make the deadline --- the only way to go fast --- is to keep the code as clean as possible at all times.
}
\quotes[Edsger Dijkstra] {
	If debugging is the process of removing software bugs, then programming must be the process of putting them in.
}
\quotes[Srinivasa Ramanujan] {
	"No," he replied, "it is a very interesting number; it is the smallest number expressible as the sum of two cubes in two different ways."
}
\quotes[Edsger W. Dijkstra] {
	Do only what only you can do.
}



\section{Introduction}
Here is the text of your introduction.

The number of total pulses is calculated by multiplying tuplets, beats and measures.

	\begin{equation}
	\label{A total pulses of a bar group}
	tuplets \times beats \times bars = pulses 
	\end{equation}


If four times swings is applied to the equation, 

~
\begin{equation}
\label{A total pulses of four time swing }
3 tuplets \times 4 beats \times 1 bar = 12 pulses
\end{equation}




\begin{equation}
\label{simple_equation}
\alpha = \sqrt{ \beta }
\end{equation}

\begin{equation}
\label{einstein}
e=mc^2
\end{equation}

Hello this is what  \textsuperscript{th} I say.


\subsection{Subsection Heading Here}
Write your subsection text here.

\begin{figure}
    \centering
%    \includegraphics[width=3.0in]{myfigure}
    \caption{Simulation Results}
    \label{simulationfigure}
\end{figure}

\section{Discussion}
Discuss here.

\subsection{Emphasizing the Last Note in a Note Group}

\quotes[William Shakespeare] {
	All's well that ends well.
}

\quotes[Atsushi Oka] {
	Rhythm is like your lover; you will be glad if she/he comes earlier,but you will be disappointed if she/he comes late --- but you will also get bored if it always comes in time.
}

~\newline
aa

\input{ly-generated/output.tex}










\section{Conclusion}
Write your conclusion here.

\end{document}


